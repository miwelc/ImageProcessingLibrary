\hypertarget{Imagen_8h}{
\section{Referencia del Archivo include/Imagen.h}
\label{Imagen_8h}\index{include/Imagen.h@{include/Imagen.h}}
}
Fichero de cabecera asociado a la biblioteca libImagen.a. 

{\tt \#include \char`\"{}imagenES.h\char`\"{}}\par


Dependencia gráfica adjunta para Imagen.h:

Gráfico de los archivos que directa o indirectamente incluyen a este archivo:\subsection*{Clases}
\begin{CompactItemize}
\item 
class \hyperlink{classImagen}{Imagen}
\begin{CompactList}\small\item\em Clase que almacena la información de una imagen y se encarga de su gestión. \item\end{CompactList}\end{CompactItemize}
\subsection*{Tipos definidos}
\begin{CompactItemize}
\item 
typedef unsigned char \hyperlink{Imagen_8h_0c8186d9b9b7880309c27230bbb5e69d}{byte}
\begin{CompactList}\small\item\em Tipo base de cada píxel. \item\end{CompactList}\end{CompactItemize}


\subsection{Descripción detallada}
Fichero de cabecera asociado a la biblioteca libImagen.a. 

\begin{Desc}
\item[Autor:]Miguel Cantón Cortés\end{Desc}
Implementación del TDA imagen (imagen digital de niveles de gris) 

Definición en el archivo \hyperlink{Imagen_8h-source}{Imagen.h}.

\subsection{Documentación de los tipos definidos}
\hypertarget{Imagen_8h_0c8186d9b9b7880309c27230bbb5e69d}{
\index{Imagen.h@{Imagen.h}!byte@{byte}}
\index{byte@{byte}!Imagen.h@{Imagen.h}}
\subsubsection[byte]{\setlength{\rightskip}{0pt plus 5cm}typedef unsigned char {\bf byte}}}
\label{Imagen_8h_0c8186d9b9b7880309c27230bbb5e69d}


Tipo base de cada píxel. 



Definición en la línea 14 del archivo Imagen.h.