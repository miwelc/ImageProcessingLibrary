\hypertarget{funciones_8h}{
\section{Referencia del Archivo include/funciones.h}
\label{funciones_8h}\index{include/funciones.h@{include/funciones.h}}
}
Fichero cabecera de funciones extra para la modificación de imagenes. 



Gráfico de los archivos que directa o indirectamente incluyen a este archivo:\subsection*{Funciones}
\begin{CompactItemize}
\item 
void \hyperlink{funciones_8h_07d05e6f2fc29075c5d6e4a80e167d43}{RGB2Gris} (const char $\ast$fich\_\-E, const char $\ast$fich\_\-S)
\begin{CompactList}\small\item\em Función que convierte una imagen RGB (PPM) a escala de grises (PGM). \item\end{CompactList}\item 
void \hyperlink{funciones_8h_373014cd670b242dc813db42de6c0fca}{mejorarContraste} (const char $\ast$fich\_\-E, const char $\ast$fich\_\-S, float gamma)
\begin{CompactList}\small\item\em Función que cambia el contraste de la imagen en función del parámetro gamma. \item\end{CompactList}\item 
void \hyperlink{funciones_8h_e2d666648802f8708d1f598b5e03f4fa}{ecualizar} (const char $\ast$fich\_\-E, const char $\ast$fich\_\-S)
\begin{CompactList}\small\item\em Función que realiza un ecualizado automático de la imagen. \item\end{CompactList}\item 
void \hyperlink{funciones_8h_0b20436889389e16c9d0d2663149751e}{umbralizar} (const char $\ast$fich\_\-E, const char $\ast$fich\_\-S, int \&T)
\begin{CompactList}\small\item\em Función que realiza un umbralizado automático de la imagen. \item\end{CompactList}\item 
void \hyperlink{funciones_8h_e20f9a11809d0788cf4553ce131197ad}{crearIcono} (const char $\ast$fich\_\-E, const char $\ast$fich\_\-S, int factor)
\begin{CompactList}\small\item\em Función que reduce el tamaño de una imagen en un factor \char`\"{}factor\char`\"{}. \item\end{CompactList}\item 
void \hyperlink{funciones_8h_306ced60827238400994da518b81f031}{morphing} (const char $\ast$fich\_\-E, const char $\ast$fich\_\-S, const char $\ast$prefijo)
\begin{CompactList}\small\item\em Función que genera una lista de imagenes interpoladas entre dos de referencia. \item\end{CompactList}\end{CompactItemize}


\subsection{Descripción detallada}
Fichero cabecera de funciones extra para la modificación de imagenes. 

\begin{Desc}
\item[Autor:]Miguel Cantón Cortés\end{Desc}
Estas funciones sirven de ejemplo de la clase \hyperlink{classImagen}{Imagen} 

Definición en el archivo \hyperlink{funciones_8h-source}{funciones.h}.

\subsection{Documentación de las funciones}
\hypertarget{funciones_8h_e20f9a11809d0788cf4553ce131197ad}{
\index{funciones.h@{funciones.h}!crearIcono@{crearIcono}}
\index{crearIcono@{crearIcono}!funciones.h@{funciones.h}}
\subsubsection[crearIcono]{\setlength{\rightskip}{0pt plus 5cm}void crearIcono (const char $\ast$ {\em fich\_\-E}, \/  const char $\ast$ {\em fich\_\-S}, \/  int {\em factor})}}
\label{funciones_8h_e20f9a11809d0788cf4553ce131197ad}


Función que reduce el tamaño de una imagen en un factor \char`\"{}factor\char`\"{}. 

\begin{Desc}
\item[Parámetros:]
\begin{description}
\item[{\em fich\_\-E}]dirección de la imagen origen \item[{\em fich\_\-S}]dirección de la imagen destino \item[{\em factor}]factor de reducción de la imagen\end{description}
\end{Desc}
Agrupa la imagen en bloques de factor$\ast$factor píxeles y calcula su media. Esta media se corresponde con cada píxel de la imagen reducida.

\begin{Desc}
\item[Parámetros:]
\begin{description}
\item[{\em fich\_\-E}]dirección de la imagen origen \item[{\em fich\_\-S}]dirección de la imagen destino \item[{\em factor}]factor de reducción de la imagen \end{description}
\end{Desc}


Definición en la línea 173 del archivo funciones.cpp.

Hace referencia a Imagen::asigna\_\-pixel(), Imagen::cargarPGM(), Imagen::guardarPGM(), Imagen::num\_\-columnas(), Imagen::num\_\-filas(), Imagen::Reserva(), y Imagen::valor\_\-pixel().

Referenciado por main().\hypertarget{funciones_8h_e2d666648802f8708d1f598b5e03f4fa}{
\index{funciones.h@{funciones.h}!ecualizar@{ecualizar}}
\index{ecualizar@{ecualizar}!funciones.h@{funciones.h}}
\subsubsection[ecualizar]{\setlength{\rightskip}{0pt plus 5cm}void ecualizar (const char $\ast$ {\em fich\_\-E}, \/  const char $\ast$ {\em fich\_\-S})}}
\label{funciones_8h_e2d666648802f8708d1f598b5e03f4fa}


Función que realiza un ecualizado automático de la imagen. 

\begin{Desc}
\item[Parámetros:]
\begin{description}
\item[{\em fich\_\-E}]dirección de la imagen origen \item[{\em fich\_\-S}]dirección de la imagen destino\end{description}
\end{Desc}
Calculamos el numero de apariciones de cada nivel de gris y lo almacenamo en el vector histograma. Calculamos las probabilidades de cada nivel de gris, dividiendo su número de apariciones entre el total de píxeles, y lo almacenamos en el vector probabilidades. Finalmente usando el vector de probabilidades, y aplicando una fórmula, obtenemos un nuevo vector con los nuevos valores de grises, transformacion

\begin{Desc}
\item[Parámetros:]
\begin{description}
\item[{\em fich\_\-E}]dirección de la imagen origen \item[{\em fich\_\-S}]dirección de la imagen destino \end{description}
\end{Desc}


Definición en la línea 72 del archivo funciones.cpp.

Hace referencia a Imagen::asigna\_\-pixel(), Imagen::cargarPGM(), Imagen::guardarPGM(), Imagen::num\_\-columnas(), Imagen::num\_\-filas(), y Imagen::valor\_\-pixel().

Referenciado por main().\hypertarget{funciones_8h_373014cd670b242dc813db42de6c0fca}{
\index{funciones.h@{funciones.h}!mejorarContraste@{mejorarContraste}}
\index{mejorarContraste@{mejorarContraste}!funciones.h@{funciones.h}}
\subsubsection[mejorarContraste]{\setlength{\rightskip}{0pt plus 5cm}void mejorarContraste (const char $\ast$ {\em fich\_\-E}, \/  const char $\ast$ {\em fich\_\-S}, \/  float {\em gamma})}}
\label{funciones_8h_373014cd670b242dc813db42de6c0fca}


Función que cambia el contraste de la imagen en función del parámetro gamma. 

\begin{Desc}
\item[Parámetros:]
\begin{description}
\item[{\em fich\_\-E}]dirección de la imagen origen \item[{\em fich\_\-S}]dirección de la imagen destino \item[{\em gamma}]Obtenemos el nuevo valor de cada píxel mediante la transformación f(x) = 255 $\ast$ (x/255)$^\wedge$gamma\item[{\em fich\_\-E}]dirección de la imagen origen \item[{\em fich\_\-S}]dirección de la imagen destino \item[{\em gamma}]\end{description}
\end{Desc}


Definición en la línea 50 del archivo funciones.cpp.

Hace referencia a Imagen::asigna\_\-pixel(), Imagen::cargarPGM(), Imagen::guardarPGM(), Imagen::num\_\-columnas(), Imagen::num\_\-filas(), y Imagen::valor\_\-pixel().

Referenciado por main().\hypertarget{funciones_8h_306ced60827238400994da518b81f031}{
\index{funciones.h@{funciones.h}!morphing@{morphing}}
\index{morphing@{morphing}!funciones.h@{funciones.h}}
\subsubsection[morphing]{\setlength{\rightskip}{0pt plus 5cm}void morphing (const char $\ast$ {\em fich\_\-E}, \/  const char $\ast$ {\em fich\_\-S}, \/  const char $\ast$ {\em prefijo})}}
\label{funciones_8h_306ced60827238400994da518b81f031}


Función que genera una lista de imagenes interpoladas entre dos de referencia. 

\begin{Desc}
\item[Parámetros:]
\begin{description}
\item[{\em fich\_\-E}]dirección de la imagen A \item[{\em fich\_\-S}]dirección de la imagen B \item[{\em prefijo}]dirección de las imagenes destino\end{description}
\end{Desc}
Va estabilizando una imagen en relación a otra sumándole o restándole en cada ciclo 1 a cada píxel hasta llegar a los mimos niveles de gris de la segunda imagen. En cada ciclo guarda cada imagen intermedia

\begin{Desc}
\item[Parámetros:]
\begin{description}
\item[{\em fich\_\-E}]dirección de la imagen A \item[{\em fich\_\-S}]dirección de la imagen B \item[{\em prefijo}]dirección de las imagenes destino \end{description}
\end{Desc}


Definición en la línea 209 del archivo funciones.cpp.

Hace referencia a Imagen::asigna\_\-pixel(), Imagen::cargarPGM(), Imagen::guardarPGM(), Imagen::num\_\-columnas(), Imagen::num\_\-filas(), y Imagen::valor\_\-pixel().

Referenciado por main().\hypertarget{funciones_8h_07d05e6f2fc29075c5d6e4a80e167d43}{
\index{funciones.h@{funciones.h}!RGB2Gris@{RGB2Gris}}
\index{RGB2Gris@{RGB2Gris}!funciones.h@{funciones.h}}
\subsubsection[RGB2Gris]{\setlength{\rightskip}{0pt plus 5cm}void RGB2Gris (const char $\ast$ {\em fich\_\-E}, \/  const char $\ast$ {\em fich\_\-S})}}
\label{funciones_8h_07d05e6f2fc29075c5d6e4a80e167d43}


Función que convierte una imagen RGB (PPM) a escala de grises (PGM). 

\begin{Desc}
\item[Parámetros:]
\begin{description}
\item[{\em fich\_\-E}]dirección de la imagen origen \item[{\em fich\_\-S}]dirección de la imagen destino\end{description}
\end{Desc}
Cada 3 bytes de la imagen PPM se corresponden a uno en la PGM, para pasar de uno a otro usamos una sencilla función: f(r,g,b) = 0.2989$\ast$r + 0.587$\ast$g + 0.114$\ast$b

\begin{Desc}
\item[Parámetros:]
\begin{description}
\item[{\em fich\_\-E}]dirección de la imagen origen \item[{\em fich\_\-S}]dirección de la imagen destino \end{description}
\end{Desc}


Definición en la línea 25 del archivo funciones.cpp.

Hace referencia a Imagen::asigna\_\-pixel(), Imagen::guardarPGM(), IMG\_\-PPM, LeerImagenPPM(), LeerTipoImagen(), y Imagen::Reserva().

Referenciado por main().\hypertarget{funciones_8h_0b20436889389e16c9d0d2663149751e}{
\index{funciones.h@{funciones.h}!umbralizar@{umbralizar}}
\index{umbralizar@{umbralizar}!funciones.h@{funciones.h}}
\subsubsection[umbralizar]{\setlength{\rightskip}{0pt plus 5cm}void umbralizar (const char $\ast$ {\em fich\_\-E}, \/  const char $\ast$ {\em fich\_\-S}, \/  int \& {\em T})}}
\label{funciones_8h_0b20436889389e16c9d0d2663149751e}


Función que realiza un umbralizado automático de la imagen. 

\begin{Desc}
\item[Parámetros:]
\begin{description}
\item[{\em fich\_\-E}]dirección de la imagen origen \item[{\em fich\_\-S}]dirección de la imagen destino\end{description}
\end{Desc}
Una vez calculado el umbral T, recorre la imagen, convirtiendo en 255 aquellos píxeles con un gris $>$= que T, y a 0 aquellos con un gris $<$ T

\begin{Desc}
\item[Parámetros:]
\begin{description}
\item[{\em fich\_\-E}]dirección de la imagen origen \item[{\em fich\_\-S}]dirección de la imagen destino \end{description}
\end{Desc}


Definición en la línea 143 del archivo funciones.cpp.

Hace referencia a Imagen::asigna\_\-pixel(), calcularUmbral(), Imagen::cargarPGM(), Imagen::guardarPGM(), Imagen::num\_\-columnas(), Imagen::num\_\-filas(), y Imagen::valor\_\-pixel().

Referenciado por main().